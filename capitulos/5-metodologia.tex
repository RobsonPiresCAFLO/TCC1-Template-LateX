% ----------------------------------------------------------
% Conclusão
% ----------------------------------------------------------
\chapter{Procedimentos metodológicos ou material e métodos}

Aqui são descritos os passos metodológicos percorridos pela pesquisa.

\section{Modalidade de pesquisa}

Quanto ao propósito: pesquisa exploratória, descritiva ou explicativa. Quanto
aos métodos empregados: segundo a natureza dos dados (quantitativa, qualitativa ou
qualiquantitativa).

Informar o ambiente em que os dados foram coletados (pesquisa de campo,
laboratório). O tipo de pesquisa: bibliográfica, documental, experimental, ensaio
clínico, estudo de coorte, caso-controle, estudo de caso, etnográfica, pesquisa-ação,
pesquisa participante, fenomenológica e mista \cite{araujo2012}.

A escolha do procedimento metodológico tem relação direta com a necessidade de atingir os objetivos da pesquisa.

\section{Instrumentos de coleta de dados}

Informar quais são os instrumentos de coleta de dados (entrevista, aplicação
de questionário, levantamento documental etc.). Como se dará a análise dos dados?
Aqui são descritos os passos metodológicos percorridos pela pesquisa \cite{IFPI_manual}.

texto texto texto texto texto texto texto texto texto texto texto texto texto texto texto 
texto texto texto texto texto texto texto texto texto texto texto texto texto texto texto 
texto texto texto.

\section{Participantes da pesquisa}

Deve informar quem são os partícipes da pesquisa, deixando claros os motivos
para a escolha daquele público.

\subsection{Aspectos éticos da pesquisa}

Avaliar e apresentar as questões éticas que envolvem a realização da
pesquisa. Deve informar se há e quais são os riscos na realização da pesquisa para
o público participante. Ex.: riscos ergonômicos, psicológicos, físicos etc.
