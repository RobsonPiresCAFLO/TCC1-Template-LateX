

\chapter{Referencial Teórico}

Destina-se a apresentar as leituras e os fundamentos teóricos que embasam a proposta da pesquisa. O pesquisador deve se atentar para o fato de que citação, conforme ABNT 10520/2023, de até três linhas não recebe recuo, às maiores que isso deve ser dado um recuo de 4 cm da margem esquerda do texto.

\section{Sessão de primeiro nível}

Texto texto texto texto texto texto texto texto texto texto texto texto texto texto texto texto texto texto texto texto texto texto texto texto texto texto texto texto texto texto texto texto texto texto texto texto texto texto texto texto texto texto texto texto.

\subsection{Título da sessão secundária}

Vale ressaltar que caso a citação seja oriunda de um trabalho em língua estrangeira, cuja tradução foi feita pelo autor do TCC, essa informação deve vir também dentro dos parênteses por meio da expressão “tradução nossa”, conforme exemplificado a seguir: “citação traduzida citação traduzida citação traduzida citação traduzida citação traduzida.” (Autor, ano, p. xx, tradução nossa).
Nos casos em que a fonte consultada da citação não possui paginação, a nota de rodapé deve trazer a seguinte informação: A fonte consultada não é paginada. Nesse caso, se enquadram informações extraídas de sites do governo, por exemplo, conforme exemplificado na NBR 10520/2023, item 7.1, exemplo 6: Exemplo 6: “O Poder Executivo envidará esforços no sentido de antecipar a entrega do Plano previsto no caput deste artigo em pelo menos 15 dias (Brasil, 1999). Nota: A fonte consultada neste exemplo 6 não é paginada.

\begin{figure}[htb]
	\caption{\label{fig_grafico}Gráfico produzido em Excel e salvo como PDF}
	\begin{center}
	    \includegraphics[scale=0.5]{imagens/abntex2-modelo-img-grafico.pdf}
	\end{center}
	\legend{Fonte: \citeonline[p. 24]{araujo2012}}
\end{figure}



\emph{Minipages} são usadas para inserir textos ou outros elementos em quadros
com tamanhos e posições controladas. Veja o exemplo da
\autoref{fig_minipage_imagem1} e da \autoref{fig_minipage_grafico2}.

\begin{figure}[htb]
 \label{teste}
 \centering
  \begin{minipage}{0.4\textwidth}
    \centering
    \caption{Imagem 1 da minipage} \label{fig_minipage_imagem1}
    \includegraphics[scale=0.2]{imagens/ifpi.pdf}
    \legend{Fonte: Produzido pelos autores}
  \end{minipage}
  \hfill
  \begin{minipage}{0.4\textwidth}
    \centering
    \caption{Grafico 2 da minipage} \label{fig_minipage_grafico2}
    \includegraphics[scale=0.2]{imagens/abntex2-modelo-img-grafico.pdf}
    \legend{Fonte: \citeonline[p. 24]{araujo2012}}
  \end{minipage}
\end{figure}

Observe que, segundo a \citeonline[seções 4.2.1.10 e 5.8]{NBR14724:2011}, as
ilustrações devem sempre ter numeração contínua e única em todo o documento.

\begin{citacao}
Qualquer que seja o tipo de ilustração, sua identificação aparece na parte
superior, precedida da palavra designativa (desenho, esquema, fluxograma,
fotografia, gráfico, mapa, organograma, planta, quadro, retrato, figura,
imagem, entre outros), seguida de seu número de ordem de ocorrência no texto,
em algarismos arábicos, travessão e do respectivo título. Após a ilustração, na
parte inferior, indicar a fonte consultada (elemento obrigatório, mesmo que
seja produção do próprio autor), legenda, notas e outras informações
necessárias à sua compreensão (se houver). A ilustração deve ser citada no
texto e inserida o mais próximo possível do trecho a que se
refere. \cite[seções 5.8]{NBR14724:2011}
\end{citacao}



