% ----------------------------------------------------------
% Metodologia
% ----------------------------------------------------------
\chapter{Objetivos}

Os objetivos da sua pesquisa devem ser divididos em objetivo geral e específicos.


% ---
\section{Objetivo Geral}

O objeto geral da pesquisa se refere ao tema ou assunto que está sendo investigado. É a questão principal que orienta toda a investigação e define os limites e escopo do estudo. Em outras palavras, o objeto geral da pesquisa é o foco central da investigação, o ponto de partida para a definição dos objetivos específicos e das hipóteses a serem testadas.
% ---
\section{Objetivos Específicos}

Os objetivos específicos da pesquisa são metas ou etapas intermediárias que precisam ser cumpridas para alcançar o seu objetivo geral. Eles detalham e descrevem as ações que serão realizadas durante a pesquisa, e servem como guia para o pesquisador na condução do estudo. Os objetivos específicos devem ser claros, precisos e mensuráveis, e devem estar diretamente relacionados ao objetivo geral da pesquisa. É comum que cada objetivo específico tenha uma seção específica no trabalho.

\index{tabelas}A \autoref{tab-nivinv} é um exemplo de tabela construída em
\LaTeX.

\begin{table}[htb]
\ABNTEXfontereduzida
\caption[Níveis de investigação]{Níveis de investigação.}
\label{tab-nivinv}
\begin{tabular}{p{2.6cm}|p{6.0cm}|p{2.25cm}|p{3.40cm}}
  %\hline
   \textbf{Nível de Investigação} & \textbf{Insumos}  & \textbf{Sistemas de Investigação}  & \textbf{Produtos}  \\
    \hline
    Meta-nível & Filosofia\index{filosofia} da Ciência  & Epistemologia &
    Paradigma  \\
    \hline
    Nível do objeto & Paradigmas do metanível e evidências do nível inferior &
    Ciência  & Teorias e modelos \\
    \hline
    Nível inferior & Modelos e métodos do nível do objeto e problemas do nível inferior & Prática & Solução de problemas  \\
   % \hline
\end{tabular}
\legend{Fonte: \citeonline{van86}}
\end{table}

Já a \autoref{tabela-ibge} apresenta uma tabela criada conforme o padrão do
\citeonline{ibge1993} requerido pelas normas da ABNT para documentos técnicos e
acadêmicos.

\begin{table}[htb]
\IBGEtab{%
  \caption{Um Exemplo de tabela alinhada que pode ser longa
  ou curta, conforme padrão IBGE.}%
  \label{tabela-ibge}
}{%
  \begin{tabular}{ccc}
  \toprule
   Nome & Nascimento & Documento \\
  \midrule \midrule
   Maria da Silva & 11/11/1111 & 111.111.111-11 \\
  \midrule 
   João Souza & 11/11/2111 & 211.111.111-11 \\
  \midrule 
   Laura Vicuña & 05/04/1891 & 3111.111.111-11 \\
  \bottomrule
\end{tabular}%
}{%
  \fonte{Produzido pelos autores.}%
  \nota{Esta é uma nota, que diz que os dados são baseados na
  regressão linear.}%
  \nota[Anotações]{Uma anotação adicional, que pode ser seguida de várias
  outras.}%
  }
  \end{table}